%%%%%%%%%%%%%%%%%%%%%%%%%%%%%%%%%%%%%%%%%%%%%%%%%%%%%%
\documentclass [11pt]{article}
\usepackage{longtable}
\usepackage{amsmath,float,epsfig,amssymb,graphicx}
\usepackage{fancyhdr,subfigure}
\usepackage{epstopdf}
\usepackage[colorlinks=true, urlcolor=blue]{hyperref}
\usepackage{titlesec}
\usepackage{color,soul}
\setcounter{secnumdepth}{3}
\usepackage{gensymb}
\usepackage{pdfpages}
\usepackage{listings}
\usepackage{enumitem}


%This commands adjust the space left for the margins
\addtolength{\oddsidemargin}{-1.0in}
\addtolength{\textwidth}{2.0in}
\addtolength{\voffset}{-0.75in}
\addtolength{\headsep}{5pt}
\addtolength{\textheight}{1.25in}
\setlength{\headheight}{14pt}

%Author & Project/Homework names
\newcommand{\authorname}{Student Name}
\newcommand{\datenumber}{October 16, 2024}
\newcommand{\assignmentID}{Homework 2}
\newcommand{\coursename}{Principles of Robot Autonomy I: }

%%%%%%%%%%%%%%%%%% DOCUMENT HEADER & TITLE %%%%%%%%%%%%%%%%%%
\fancyhead{}
\fancyhead[C]{\large \coursename \assignmentID \ | \datenumber \ | \authorname}

\begin{document}
\pagestyle{fancy}

\begin{center}
    \Large \textbf{\coursename \assignmentID}\\
    \small \authorname\\
    \small \datenumber
\end{center}

%%%%%%%%%%%%%%%% EXAM %%%%%%%%%%%%%%%%%%%%

\section*{Problem 1}
    \begin{enumerate}[label=(\roman*)]
        \item \textbf{Transcription to Finite Dimensional Optimization Problem}

        To transcribe this optimal control problem into a finite-dimensional constrained optimization problem, we use direct transcription with discretization. Let's discretize the time interval $[0, t_f]$ into $N$ segments with time step $\Delta t$, where $t_i = i \cdot \Delta t$ for $i = 0, 1, \ldots, N$, and $t_N = t_f$.

        \textbf{Decision Variables:}

        The optimization variables are:
        \begin{equation}
        \mathbf{z} = [x_0, y_0, \theta_0, \ldots, x_N, y_N, \theta_N, v_0, \omega_0, \ldots, v_{N-1}, \omega_{N-1}, t_f]
        \end{equation}

        This gives us $3(N+1) + 2N + 1 = 5N + 4$ decision variables.

        \textbf{Objective Function:}

        The continuous cost functional is discretized using numerical integration (e.g., trapezoidal rule or rectangle rule):
        \begin{equation}
        \min_{\mathbf{z}} \quad J = \sum_{i=0}^{N-1} \left(\alpha + v_i^2 + \omega_i^2\right) \Delta t
        \end{equation}

        where $\Delta t = t_f / N$.

        \textbf{Constraints:}

        \textit{1. Initial Conditions:}
        \begin{align}
        x_0 &= 0 \\
        y_0 &= 0 \\
        \theta_0 &= \pi/2
        \end{align}

        \textit{2. Final Conditions:}
        \begin{align}
        x_N &= 5 \\
        y_N &= 5 \\
        \theta_N &= \pi/2
        \end{align}

        \textit{3. Dynamics Constraints (using forward Euler integration):}

        For $i = 0, 1, \ldots, N-1$:
        \begin{align}
        x_{i+1} &= x_i + v_i \cos(\theta_i) \Delta t \\
        y_{i+1} &= y_i + v_i \sin(\theta_i) \Delta t \\
        \theta_{i+1} &= \theta_i + \omega_i \Delta t
        \end{align}

        \textit{4. Collision Avoidance Constraints:}

        For $i = 0, 1, \ldots, N$:
        \begin{equation}
        \sqrt{(x_i - 2.5)^2 + (y_i - 2.5)^2} - 0.4 \geq 0
        \end{equation}

        where $0.4 = r_{ego} + r_{obstacle} = 0.1 + 0.3$.

        \textbf{Complete Optimization Problem:}

        \begin{align}
        \min_{\mathbf{z}} \quad & \sum_{i=0}^{N-1} \left(\alpha + v_i^2 + \omega_i^2\right) \frac{t_f}{N} \\
        \text{subject to} \quad & x_0 = 0, \quad y_0 = 0, \quad \theta_0 = \pi/2 \\
        & x_N = 5, \quad y_N = 5, \quad \theta_N = \pi/2 \\
        & x_{i+1} = x_i + v_i \cos(\theta_i) \frac{t_f}{N}, \quad i = 0, \ldots, N-1 \\
        & y_{i+1} = y_i + v_i \sin(\theta_i) \frac{t_f}{N}, \quad i = 0, \ldots, N-1 \\
        & \theta_{i+1} = \theta_i + \omega_i \frac{t_f}{N}, \quad i = 0, \ldots, N-1 \\
        & \sqrt{(x_i - 2.5)^2 + (y_i - 2.5)^2} \geq 0.4, \quad i = 0, \ldots, N \\
        & t_f > 0
        \end{align}

        \item \textbf{Implementation Approach}

        The implementation of the trajectory optimization uses the direct transcription method with \texttt{scipy.optimize.minimize}. The key components are:

        \textbf{1. Parameterization:} The trajectory is discretized into $N$ time steps. The decision variables include the state at each time step $(x_i, y_i, \theta_i)$ and controls $(v_i, \omega_i)$, along with the final time $t_f$.

        \textbf{2. Objective Function:} The cost is computed as:
        \begin{equation}
        J = \sum_{i=0}^{N-1} \left(\alpha + v_i^2 + \omega_i^2\right) \Delta t
        \end{equation}

        \textbf{3. Constraint Handling:} Constraints are implemented as:
        \begin{itemize}
            \item Equality constraints for boundary conditions (initial and final states)
            \item Equality constraints for dynamics (discretized using Euler integration)
            \item Inequality constraints for collision avoidance at each discretization point
        \end{itemize}

        \textbf{4. Optimization:} The \texttt{scipy.optimize.minimize} function with the SLSQP (Sequential Least Squares Programming) method is used to solve the constrained nonlinear optimization problem. This method handles both equality and inequality constraints efficiently.

        \textbf{5. Initial Guess:} A good initial guess is crucial for convergence. Typically, a straight-line path in configuration space with constant controls provides a reasonable starting point.

        The optimized trajectory successfully navigates from the start to the goal while avoiding the obstacle at $(2.5, 2.5)$ with radius $0.3$.

        [Include trajectory plot here from the notebook]

        \item \textbf{Effect of Different $\alpha$ Values}

        The parameter $\alpha$ in the cost function $J = \int_0^{t_f} (\alpha + v^2 + \omega^2) dt$ balances time optimality versus control effort:

        \textbf{Small $\alpha$ (e.g., $\alpha = 0.1$):}
        \begin{itemize}
            \item The optimizer prioritizes minimizing control effort $(v^2 + \omega^2)$ over time
            \item Results in slower, smoother trajectories with smaller velocities and angular rates
            \item The robot takes longer to reach the goal but uses less aggressive control actions
            \item The trajectory tends to follow a more conservative path around the obstacle
        \end{itemize}

        \textbf{Medium $\alpha$ (e.g., $\alpha = 1.0$):}
        \begin{itemize}
            \item Provides a balanced trade-off between time and control effort
            \item The trajectory completes in moderate time with reasonable control magnitudes
            \item Represents a practical compromise for real robot systems
        \end{itemize}

        \textbf{Large $\alpha$ (e.g., $\alpha = 10.0$):}
        \begin{itemize}
            \item The optimizer strongly prioritizes minimizing time $t_f$
            \item Results in faster trajectories with larger control inputs
            \item The robot reaches the goal quickly but with more aggressive maneuvers
            \item May lead to higher velocities and sharper turns around the obstacle
            \item The trajectory becomes closer to minimum-time control
        \end{itemize}

        \textbf{Summary:} As $\alpha$ increases, the weight on time increases relative to control effort, leading to faster but more aggressive trajectories. Conversely, smaller $\alpha$ values produce slower, smoother paths that conserve energy. The choice of $\alpha$ should reflect the specific application requirements: use larger values when time is critical, and smaller values when smooth, energy-efficient motion is preferred.

    \end{enumerate}

\section*{Problem 2}
    \begin{enumerate}[label=(\roman*)]
        \item
        \item
        \item
    \end{enumerate}

\section*{Problem 3}
    \begin{enumerate}[label=(\roman*)]
        \item
        \item
        \item
        \item
    \end{enumerate}

\section*{Problem 4}
    Note: This problem is not graded but should be completed for section preparation.
    \begin{enumerate}[label=(\roman*)]
        \item
        \item
    \end{enumerate}



%%%%%%%%%%%%%%%%%% TEMPLATE FOR CODE SUBMISSION %%%%%%%%%%%%%%%%%%
\newpage
\section*{Appendix A: Code Submission Template}
    \begin{lstlisting}[language=Python]
        PASTE CODE HERE
    \end{lstlisting}

%%%%%%%%%%%%%%%%%% TEMPLATE FOR IMAGE SUBMISSION %%%%%%%%%%%%%%%%%%
% \section*{Appendix B: Image Submission Template}
%     \begin{figure}[h]
%         \centering
%         \includegraphics[width=0.5\textwidth]{figs/example_image.jpg}
%         \caption{INSERT CAPTION HERE}
%         \label{fig:placeholder}
%     \end{figure}
    
    

\end{document}
